\begin{center}
\begin{abstract}
	In the framework of our Effective-Programming Course, we explore the sensitivity of Intent-to-Treat estimates to different imputation methods using real data from the \ac{GATE} experiment, a longitudinal study conducted by the US Department of Labor in which free entrepreneurship training was randomly offered to individuals interested in starting or running a business. We contrast the results of the complete-case analysis with four other scenarios in which we complete the missing data in the outcome of interest and covariates with the k-Nearest-Neighbor imputation method, a random draw from a normal distribution. We also do a bound analysis, by imputing the minimum or maximum value for the outcome of interest. The results show that the understanding of the missing pattern is crucial for the interpretation of the results, based on the imputed data. This project uses the template by \cite{GaudeckerEconProjectTemplates}.

\end{abstract}
\end{center}