% !TEX root = 000_paper.tex

\section{Introduction}

% MOTIVATION:

% Problematic RCTs and missing data
\acp{RCT} are widely used in economics, especially in the context of policy evaluation. In the ideal case, the random assignment of the population of interest into treatment and control groups eliminates the selection bias. For this reason the \ac{RCT} is sometimes referred to as the "golden standard'' of the evaluation methods.
However, its validity can be threatened by missing data due to non-response or attrition in the follow up surveys(see\cite{angristmostly2008}). Incomplete data becomes a problem in case there is evidence that observations are missing in a non-random way, because the remaining sample might not be representative. Individuals who are not willing to provide relevant information or participate in all follow-up surveys might be systematically different from the individuals who provide full information and participate in all follow up surveys. Consequently, handling missing data inappropriately might lead to biased results \cite{heckman1976common, lee2009training, kling2007experimental}.


% TYPE OF MISSINGS IN THE LITERATURE
\cite{little1989analysis} differentiate between three missing data mechanisms: data missing at completely random \ac{MCAR}, data missing at random \ac{MAR}, and data missing at non random \ac{MANR}. \ac{MCAR} is unrelated to observed and unobserved characteristics, \ac{MAR} is related to observed characteristics only, and \ac{MANR} is related to both observed and/ or unobserved characteristics. Information on the missing data mechanism can be inferred from observed data, yet, assumptions regarding the unobserved characteristics remain untestable.

% WHY SENSITIVITY ANALYSIS
Based on the underlying mechanism there are different ways to handle missing data. If the assumed underlying missing data mechanism is incorrect, the result may still be biased (Li et al., 2014). For example, the complete-case analysis is the easiest way to handle missing data; however, it is likely to be biased if the underlying mechanism is not\ac{MCAR}. In order to produce robust results, a sensitivity analysis which varies the assumptions about the underlying missing data mechanism and employs the appropriate methods respectively is necessary.

% OUR ANALYSIS IN DETAIL
In our work we explore the sensitivity of \ac{ITT} estimates to different imputation methods using real data from the \ac{GATE} experiment. \ac{GATE}was a subsidized entrepreneurship training program which run between 2003 and 2003 in the US. Please see Fairlie et al. (2015) for more information and a broader analysis of \ac{GATE}.

We contrast \ac{OLS} regression outcomes based on the complete data set with four other scenarios in which we complete the missing data in the outcome and covariates. To do so, we impute missing data in four different ways: (1) Based on the analysis of the missings, we assume that the underlying mechanism for the missings is following the \ac{MAR} and complete missings in the outcome and covariates with the \ac{kNN} imputation method. (2) We impute the missings in the covariates with the \ac{kNN} imputation method while imputing the missings in the outcomes with a random draw from a specified normal distribution. Finally, we show the extreme outcomes by imputing the missings in the outcome by the (3) minimum and the (4) maximum value, respectively.

% RESULTS
Comparing the \ac{OLS} regression estimates resulting from (1)-(4) to the complete-case analysis, we find that the estimated treatment coefficient using the \ac{kNN} imputation method (1) only exposes a negligible change. Further, the treatment effect decreases when using the imputation method (2). The treatment effect estimates from the complete-case and the imputation methods (1) and (2), fall within the bounds of the imputation methods (3) and (4). All in all, we confirm that an analysis on the underlying missing mechanism is crucial to gauge the missing mechanism.

% END
The rest of this work is structured as follows: In Section~\ref{sec:GATE} we introduce the \ac{GATE} project, in Section~\ref{sec:missing_analysis} we analyze if the missings expose any patterns, in Section~\ref{sec:imputation_method} we explain the imputation methods in more details and establish hypothesis about the estimation outcomes, in Section~\ref{sec:itt_estimates} we present the result, in Section~\ref{sec:conlcusion} we conclude.
