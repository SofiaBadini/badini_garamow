% !TEX root = 000_paper.tex

\section{The GATE Experiment}

Project Growing America Through Entrepreneurship (GATE) was a longitudinal study conducted by the US Department of Labor and the Small Business Administration in which free entrepreneurship training was randomly offered to individuals interested in starting or running a business. More than 4000 individuals applied for a limited number of slots, making this, as of 2015, the largest-ever randomized evaluation of entrepreneurship training and assistance.

Project GATE sought to increase employment, earnings, and self-sufficiency by promoting economic development in low-income areas. Most communities have organizations that provide assistance to people who want to start their own businesses, but Project GATE differed from the programs already available at the sites: it employed a particularly extensive outreach, it provided highly individualized training, it was free and it did not screen out participants based on their likelihood of success.

Subjects assigned to the treatment group were offered an array of (highly heterogeneous) free best-practice training services, while subjects assigned to the control group were not offered any free service. The treatment phase ran from September 2003 to July 2005 in 7 different sites across 3 states and, after the application survey, there were follow-up surveys at 6, 18, and 60 months after random assignment.

The applicants' answers to the application survey provide a rich set of baseline characteristics. These include general information about age, gender, racial background and education; as well as information about household composition and socio-economics status, health, credit history, and employment. Personal characteristics associated with entrepreneurial success, which can be boiled down to two main (standardized) indexes of risk-tolerance and autonomy, were also assessed via these self-reported surveys.

Later follow-up surveys asked whether the individuals received GATE services since random assignment and how they evaluated the usefulness of those services; and contained questions on measures of business outcomes such as business ownership, business size and revenues, household income, and overall work satisfaction.

Fairlie, Kalran, and Zinman (2015) estimate the effects of receiving entrepreneurship training via GATE services (LATE). First, they check whether participating into the GATE experiment actually increased the quantity and quality of training received by the treated. Second, they estimate the impact of such training on a number of business outcomes at each time horizon, restricting the anaylsis to the complete case.

Their results suggest that entrepreneurship training had limited impacts on business ownership, scale, and household income. While entrepreneurship training did increase the likelihood of business ownership in the short-run (6-months), this effect depreciated over time and was not significant anymore at 18 and 60 months. There is no evidence that training affected other outcomes at any horizon.

In this paper, we focus on the effects of being offered free entrepreneurship training ("intent-to-treat" effect, or ITT) on log household income 18 months after random assignment.

In principle, if entrepreneurship training is an investment in human capital (or functions as a signaling device on the labor market), recipients should have better financial outcomes than non-recipients, at least in the medium run; either because they are more likely to successfully start or run a business or because they are able to find or transition to better jobs. On the other hand, if recipients decide to invest their own savings in starting a business, this will have a negative impact on their household income. In general, the impact of being taught entrepreneurial skills is unclear, at least in developing countries (\cite{mckenzie2014}).
