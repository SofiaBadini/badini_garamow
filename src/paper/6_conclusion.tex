% !TEX root = 000_paper.tex

\section{Conclusion}
\label{sec:conlcusion}
\acp{RCT} in economics are considered as the ``golden standard''. Yet, the validity of its estimates is threatened if the data is incomplete and the missings expose a pattern. The data may expose three underlying mechanisms, data missing at completely random \ac{MCAR}, data missing at random \ac{MAR}, and data missing at non random \ac{MANR}. The underlying missing mechanism can be inferred from observed data to come up with reasonable assumptions about the missing data mechanism, yet, assumptions regarding the unobserved characteristics remain untestable.

In our work we therefore analyze the sensitivity of \ac{OLS} regression estimates, using the data from the \ac{GATE} project. Therefore, we compare the \ac{OLS} estimates from the complete analysis to the analysis based on imputed data. We impute the data in four different ways. First, we impute the missing in the outcome and covariates with the \ac{kNN} method, ``kNN scenario''. For the remaining three imputations, we complete the missing information on the outcome by imputing a random draw from a normal distribution, or imputing the minimum or maximum, respectively. The covariates for these three remaining imputation scenarios are imputed with the \ac{kNN} method, labeled as ``kNN-msd scenario'', ``kNN-min scenario'', ``kNN-max scenario'', respectively.

Based in the missing analysis, it is reasonable to assume that the underlying mechanism follows an \ac{MAR} pattern. We observe that missing values are correlated with the treatment status and income level, i.e., mostly people in the control group and people with lower income tend to deprive information.

Our analysis shows that it is important to understand the missing patterns. This understanding is crucial for the interpretation of the results, based on the different imputation methods. As expected, the ``kNN scenario'', the \ac{OLS} estimate of the treatment effect (``Treatment'') expose only a negligible change, while the standard errors decrease, mainly due to increase number of observations. The treatment effect hardly changes because the missings are imputed with information which is closely related to the characteristics of the missing. The results in the remaining scenarios are driven by the fact that most of the missings are from the control group and missingness is correlated with a lower household income. The coefficient of the treatment in the ``kNN-msd scenario'' decreases because the imputing the missing outcome with the mean plus, minus some random noise decreases the difference between the treatment and control group. The ``kNN-min scenario'' shows the upper bound of the treatment effect. The missings in the outcome are imputed with the minimum value, hence, increasing the difference between the treatment and control. The opposite is true for the ``kNN-max scenario'', the imputation of missings in the outcome with a maximum increase the difference between the treatment and control. The contract between the treatment and control becomes so evident that the treatment effect turns significant in the ``kNN-min scenario'' and ``kNN-max scenario''. This bound analysis is informative of the possible extreme outcomes. All estimates fall into the bounds.

All in all, we conclude that the analysis on the missing pattern is important to infer assumptions about the underlying missing mechanism. In the missing mechanism is justified, results are robust. However, if imputation methods are used without any justification for the underlying mechanism, the estimated results are sensitive. Hence, a bound analysis is an informative extension, especially when one is reluctant to make assumptions about the underlying mechanism.