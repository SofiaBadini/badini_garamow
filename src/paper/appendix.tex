% !TEX root = 000_paper.tex

\begin{appendices}
	\section*{APPENDIX}
\subsection*{Additional Tables for Missing Data Analysis} \label{appendix_missing_analysis}

\renewcommand{\thetable}{\Alph{section}\arabic{table}}
\setcounter{table}{0}

\renewcommand{\thefigure}{\Alph{section}\arabic{figure}}
\setcounter{figure}{0}

\begin{table}[H]
\centering
\caption{\textsc{Levene's Test for Homogeneity of Variances}}
\begin{adjustbox}{width=\textwidth,totalheight=\textheight,keepaspectratio}
\input{../../bld/out/tables/table_levene_df}
\end{adjustbox}


\label{tab:table_levene}
\medskip
\raggedright
\footnotesize
\textit{Notes:} Levene's tests for equal variances are performed for all the individuals' characteristics.
All reported characteristics are measured at time of application, prior to random assignment. The wave 2 survey is conducted 18 months after time of application. We adjust the significance levels for multiple testing using a Bonferroni correction. \\
*** Significant at the (adjusted) 1 percent level ** Significant at the (adjusted) 5 percent level * Significant at the (adjusted) 10 percent level
\end{table}

\newpage

\begin{table}[H]
\centering
\caption{\textsc{Missing Values Comparison of Characteristics for GATE Experiment}}
\begin{adjustbox}{width=\textwidth,totalheight=\textheight,keepaspectratio}
\input{../../bld/out/tables/table_chisq_df}
\end{adjustbox}


\label{tab:table_chisq}
\medskip
\raggedright
\footnotesize
\textit{Notes:} Chi-squared tests are performed to assess whether missingness in the data is related to individuals' characteristics.
All reported characteristics are measured at time of application, prior to random assignment. The outcome of interest is log household income 18 months after time of application. We adjust the significance levels for multiple testing using a Bonferroni correction. \\
*** Significant at the (adjusted) 1 percent level ** Significant at the (adjuted) 5 percent level * Significant at the (adjusted) 1 percent level
\end{table}
%%%%%%%%%%%%%%%%%%%%%%%%%%%%%%%%%%%%%%

\newpage
\subsection*{Additional Tables for ITT Estimates} \label{appendix_itt_estimates}

\newpage
%%%%%%%%%%%%%%%%%%%%%%%%%%%%%%%%%%%%%%%
\begin{table}[H]
\centering
\caption{\textsc{ITT Estimates on log Household Income}}
\addtolength{\tabcolsep}{24pt}
\input{../../bld/out/tables/table_gate_complete_no_controls_coeff}
\addtolength{\tabcolsep}{-24pt}

\medskip
\input{../../bld/out/tables/table_gate_complete_no_controls_summary}

\label{tab:table_complete_nocontr}
\bigskip
\raggedright
\footnotesize
\textit{Notes:} The dependent variable is log household income 18 months after random assignment. \\
\end{table}
%%%%%%%%%%%%%%%%%%%%%%%%%%%%%%%%%%%%%%%

%%%%%% ANALYSIS ON IMPUTED DATA  %%%%%%
%%%%%%%%%%%%%%%%%%%%%%%%%%%%%%%%%%%%%%%
\begin{table}[h!]
\centering
\caption{\textsc{ITT Estimates on log Household Income - Analysis on Imputed Data}}
\input{../../bld/out/tables/table_data_imputed_kNN_msd_controls_coeff}


\medskip
\begin{adjustbox}{width=0.825\textwidth, totalheight=\textheight, keepaspectratio}
\input{../../bld/out/tables/table_data_imputed_kNN_msd_controls_summary}
\end{adjustbox}

\label{tab:table_kNN_msd_withcontr}
\bigskip
\raggedright
\footnotesize
\textit{Notes:} The dependent variable is log household income 18 months after random assignment. All reported characteristics are measured at time of application, prior to random assignment. The missing values in the data are imputed as follows: First, the outcomes are imputed with the mean from the draws from a normal distribution, with the mean equal to the median and a share of 0.25 of the standard deviation of each column. In a second step, the covariates are imputed with the \ac{kNN} estimator. \\
*** Significant at the 1 percent level ** Significant at the 5 percent level * Significant at the 10 percent level
\end{table}
%%%%%%%%%%%%%%%%%%%%%%%%%%%%%%%%%%%%%%%

\newpage
%%%%%%%%%%%%%%%%%%%%%%%%%%%%%%%%%%%%%%%
\begin{table}[h!]
\centering
\caption{\textsc{ITT Estimates on log Household Income - Analysis on Imputed Data}}
\input{../../bld/out/tables/table_data_imputed_kNN_min_controls_coeff}


\medskip
\begin{adjustbox}{width=0.825\textwidth, totalheight=\textheight, keepaspectratio}
\input{../../bld/out/tables/table_data_imputed_kNN_min_controls_summary}
\end{adjustbox}

\label{tab:table_kNN_min_withcontr}
\bigskip
\raggedright
\footnotesize
\textit{Notes:} The dependent variable is log household income 18 months after random assignment. All reported characteristics are measured at time of application, prior to random assignment. The missing values in the data are imputed as follows: The outcomes are imputed with the minimum of the variable/ column while the covariates are imputed with the \ac{kNN} imputation method. \\
*** Significant at the 1 percent level ** Significant at the 5 percent level * Significant at the 10 percent level
\end{table}
%%%%%%%%%%%%%%%%%%%%%%%%%%%%%%%%%%%%%%%

\newpage
%%%%%%%%%%%%%%%%%%%%%%%%%%%%%%%%%%%%%%%
\begin{table}[H]
\centering
\caption{\textsc{ITT Estimates on log Household Income - Analysis on Imputed Data}}
\input{../../bld/out/tables/table_data_imputed_kNN_max_controls_coeff.tex}


\medskip
\begin{adjustbox}{width=0.825\textwidth, totalheight=\textheight, keepaspectratio}
\input{../../bld/out/tables/table_data_imputed_kNN_max_controls_summary.tex}
\end{adjustbox}

\label{tab:table_kNN_max_withcontr}
\bigskip
\raggedright
\footnotesize
\textit{Notes:} The dependent variable is log household income 18 months after random assignment. All reported characteristics are measured at time of application, prior to random assignment. The missing values in the data are imputed as follows: The outcomes are imputed with the maximum of the variable/ column while the covariates are imputed with the \ac{kNN} imputation method. \\
*** Significant at the 1 percent level ** Significant at the 5 percent level * Significant at the 10 percent level
\end{table}
%%%%%%%%%%%%%%%%%%%%%%%%%%%%%%%%%%%%%%%



\end{appendices}
